\subsection{Sets}
\tsDef{3.2 (Equal sets)}{
    For sets $A$ and $B$,
    \(
    A = B \;\Longleftrightarrow\; \forall x \,(x \in A \iff x \in B).
    \)
}
\newline
\tsLem{3.1 (Equality of set elemetns and ord. pairs)}{
    For any sets $A$ and $B$,
    \(
    \{A\} = \{B\} \;\Longrightarrow\; A = B.
    \)
    \\
    \textbf{Ordered pairs:}
    \(
    (a,b) = (c,d) \;\Longleftrightarrow\; a = c \;\wedge\; b = d.
    \)
    \\
    \textbf{Odered pairs via sets:}
    \(
    (a,b) := \bigl\{ \{a\}, \{a,b\} \bigr\}.
    \)
}
\newline
\tsDef{3.3 (Subset)}{
    \(
    A \subseteq B \;\Longleftrightarrow\; \forall x \,(x \in A \Rightarrow x \in B).
    \)
}
\newline
\tsLem{3.2 (Sets equality and subsets)}{
    \(
    A = B \;\Longleftrightarrow\; (A \subseteq B) \wedge (B \subseteq A).
    \)
    Equivalently: \\
    \(
    \forall x \,\bigl((x \in A \Rightarrow x \in B) \wedge (x \in B \Rightarrow x \in A)\bigr)
    \;\Leftrightarrow\;
    \forall x \,(x \in A \iff x \in B).
    \)
}
\newline
\tsLem{3.3 (Transitivity of subsets)}{
    For all sets $A,B,C$,
    \(
    A \subseteq B \;\wedge\; B \subseteq C \;\Longrightarrow\; A \subseteq C.
    \)
}
\newline
\tsDef{3.4 (Union and Intersection)}{
    \(
    A \cup B := \{x \mid x \in A \vee x \in B\},
    \)
    \(
    A \cap B := \{x \mid x \in A \wedge x \in B\}.
    \)
    \\
    \textbf{Families of Sets:}
    Let $\mathcal{A}$ be a set of sets:\\
    \(
    \bigcap \mathcal{A} := \{x \mid x \in A \text{ for all } A \in \mathcal{A}\},
    \)
    \(
    \bigcup \mathcal{A} := \{x \mid x \in A \text{ for some } A \in \mathcal{A}\}.
    \)
    \\
    \textbf{Example:}
    \\
    \(
    \mathcal{A} = \bigl\{\{a,b,c\}, \{a,c\}, \{a,b,c,f\}, \{a,c,d\}\bigr\},
    \)
    \\
    \(
    \bigcup \mathcal{A} = \{a,b,c,d,e,f\},
    \qquad
    \bigcap \mathcal{A} = \{a,c\}.
    \)
    \\
    If $I$ is an index set and $\mathcal{A} = \{A_i \mid i \in I\}$, then
    \(
    \bigcap_{i \in I} A_i,
    \
    \bigcup_{i \in I} A_i.
    \)
}
\newline
\tsDef{3.5 (Set Difference)}{
    The difference of sets $B$ and $A$ is
    \(
    B \setminus A := \{x \in B \mid x \notin A\}.
    \)
}
\newline
\tsDef{3.6 (Empty Set)}{
    A set is called \emph{empty} if it contains no elements:
    \(
    \forall x \,(x \notin A).
    \)
}
\newline
\tsLem{3.5}{
    There is exactly \textbf{one} empty set, denoted $\varnothing$ or $\{\}$.
}
\newline
\tsLem{3.6}{
    The empty set is a subset of every set:
    \(
    \forall A \;(\varnothing \subseteq A).
    \)
    \\
    \textbf{Construction of natural numbers:}
    \(
    S(n) := n \cup \{n\} \ \text{(rec. sucsessor)}.
    \)
}
\newline
\tsDef{3.7 (Power Set)}{
    The power set of a set $A$, denoted $\mathcal{P}(A)$, is the set of all subsets of $A$:
    \(
    \mathcal{P}(A) := \{S \mid S \subseteq A\}.
    \)
    If $|A| = k$, then
    \(
    |\mathcal{P}(A)| = 2^k.
    \)
    In particular, for a set with $k$ elements, each element may be
    \emph{included} or \emph{excluded}, giving
    \(
    2 \times 2 \times \cdots \times 2 = 2^k
    \)
    possible subsets.
    \\
    \textbf{Example (bit-mask intuition for $\{a,b\}$):}
    \(
    {\scriptsize
            \begin{array}{ccl}
                00 & \longleftrightarrow & \varnothing \\
                01 & \longleftrightarrow & \{b\}       \\
                10 & \longleftrightarrow & \{a\}       \\
                11 & \longleftrightarrow & \{a,b\}
            \end{array}
        }
    \)
    % \subsection*{Counting Principle}
    % If a process has:
    % \begin{itemize}[noitemsep, topsep=0pt]
    %     \item $a$ choices for step 1,
    %     \item $b$ choices for step 2,
    %     \item $c$ choices for step 3,
    % \end{itemize}
    % then the total number of outcomes is
    % \(
    % a \times b \times c.
    % \)
}
% \newline
\subsection{Relations}
\tsDef{3.8 (Cartesian product)}{
    The Cartesian product $A \times B$ of sets $A$ and $B$ is the set of all ordered pairs with first component from $A$ and second from $B$:
    \(
    A \times B := \{(a,b) \mid a \in A,\ b \in B\}.
    \)
    The cardinality satisfies
    \(
    |A \times B| = |A| \cdot |B|.
    \)
    \\
    \textbf{More generally:}
    \(
    \mathop{\scalebox{1.6}{$\times$}}_{i=1}^k A_i
    :=
    \{(a_1,\dots,a_k) \mid a_i \in A_i \text{ for } 1 \le i \le k\}.
    \)
    \\
    The Cartesian product is \emph{not associative}, since elements are ordered tuples.
    \\
    \textbf{Connection to power sets:}
    \\
    If
    \(
    A = \{a,b,c\}, \ |A|=3,
    \)
    then each element may be either \emph{in} or \emph{out}, giving
    \(
    \{0,1\}^3 = \{0,1\} \times \{0,1\} \times \{0,1\},
    \)
    which represents all subsets of $A$.
    \\
    \textbf{Another example:}
    \\
    \(
    A_1 = \{0,1\}, \ A_2 = \{d,e\},
    \)
    \(
    A_1 \times A_2 = \{(0,d),(0,e),(1,d),(1,e)\}.
    \)
}
\newline
\tsDef{3.9 (Relation)}{
    A (binary) relation $\rho$ from a set $A$ to a set $B$ is a subset of $A \times B$.
    \\
    If $A=B$, then $\rho$ is called a relation \emph{on} $A$.
    \\
    \textbf{Notation:}
    \(
    (a,b) \in \rho \;\Rightarrow\; a \,\rho\, b,
    \
    (a,b) \notin \rho \;\Rightarrow\; a \not\!\rho\, b.
    \)
    \\
    For any set $S$, any subset $\rho \subseteq S \times S$ is a relation on $S$.
    \\
    There are
    \(
    2^{n^2}
    \)
    relations on a set of cardinality $n$, since
    \(
    |S \times S| = n^2
    \ \text{and} \
    |\mathcal{P}(S \times S)| = 2^{n^2}.
    \)
    \\
    % \textbf{Relations as Matrices:}
    % \\
    % Relations can be represented by $0$–$1$ matrices, analogous to adjacency matrices for graphs. Combining relations using set operations
    % \(
    % \cap,\ \cup,\ \setminus
    % \)
    % corresponds to applying logical operations
    % \(
    % \wedge,\ \vee,\ \neg
    % \)
    % entrywise to matrices.
    % \\
    \textbf{Examples on $\mathbb{Z}$:}
    \begin{itemize}[noitemsep, topsep=-3pt]
        \item $\leq \;\cup\; \geq$ is the complete relation $\mathbb{Z} \times \mathbb{Z}$.
        \item $\leq \;\cap\; \geq$ is the identity relation:
              \(
              \{(a,a) \mid a \in \mathbb{Z}\}.
              \)
    \end{itemize}
}
\tsDef{3.11 (Inverse Relation)}{
    The inverse relation of $\rho$ is
    \(
    \rho^{-1} := \{(b,a) \mid (a,b) \in \rho\}.
    \)
    \\
    Equivalently,
    \(
    b \,\rho^{-1}\, a \;\Longleftrightarrow\; a \,\rho\, b.
    \)
    \\
    \textbf{Interpretations:}
    \begin{itemize}[noitemsep, topsep=-3pt]
        \item In graphs: reversing all edge directions.
        \item In matrices: taking the transpose.
    \end{itemize}
}
\tsDef{3.12 (Composition of Relations)}{
    Let $\rho \subseteq A \times B$ and $\sigma \subseteq B \times C$.
    The composition $\sigma \circ \rho$ is defined by
    \(
    \sigma \circ \rho
    :=
    \{(a,c) \mid \exists b \ ((a,b) \in \rho \wedge (b,c) \in \sigma)\}.
    \)
    Composition is associative:
    \(
    \rho \circ (\sigma \circ \tau) = (\rho \circ \sigma) \circ \tau.
    \)
}
\newline
\tsLem{3.8}{
    Let $\rho$ be a relation from $A$ to $B$ and $\sigma$ a relation from $B$ to $C$.
    Then
    \(
    (\sigma \circ \rho)^{-1} = \rho^{-1} \circ \sigma^{-1}.
    \)
}
\subsection{Properties of Relations}
\begin{center}
    \renewcommand{\arraystretch}{1.4}
    \begin{tabular}{lccc}
        \toprule
        \textbf{Name} & \textbf{Formula}                             & \textbf{Set} & \textbf{Example} \\
        \midrule
        Reflexive
                      & $a\rho a$
                      & $\mathrm{id} \subseteq \rho$
                      & $\mathrm{id},\, \geq$                                                          \\

        Irreflexive
                      & $\neg(a\rho a)$
                      & $\mathrm{id} \cap \rho = \varnothing$
                      & $\neq,\, >$                                                                    \\

        Symmetric
                      & $a\rho b \iff b\rho a$
                      & $\rho = \hat{\rho}$
                      & $\mathrm{id},\, \equiv\ (\mathrm{mod}\ m)$                                     \\

        Antisymmetric
                      & $a\rho b \wedge b\rho a \rightarrow a=b$
                      & $\rho \cap \hat{\rho} \subseteq \mathrm{id}$
                      & $\geq,\, \mid$                                                                 \\

        Transitive
                      & $a\rho b \wedge b\rho c \rightarrow a\rho c$
                      & $\rho^{2} \subseteq \rho$
                      & $\equiv\ (\mathrm{mod}\ m),\, >$                                               \\
        \bottomrule
    \end{tabular}
\end{center}
% \tsDef{3.13 (Reflexive Relation)}{
%     A relation $\rho$ on a set $A$ is called \emph{reflexive} if
%     \(
%     \forall a \in A,\ a \,\rho\, a.
%     \)
%     \\
%     Equivalently,
%     \(
%     \mathrm{id}_A \subseteq \rho.
%     \)
% }
% \newline
% \tsDef{3.14 (Irreflexive Relation)}{
%     A relation $\rho$ on a set $A$ is called \emph{irreflexive} if
%     \(
%     \forall a \in A,\ a \not\!\rho\, a.
%     \)
%     \\
%     Equivalently,
%     \(
%     \rho \cap \mathrm{id}_A = \varnothing.
%     \)
% }
% \newline
% \tsDef{3.15 (Symmetric Relation)}{
%     A relation $\rho$ on a set $A$ is called \emph{symmetric} if
%     \(
%     a \,\rho\, b \;\Leftrightarrow\; b \,\rho\, a
%     \ \forall a,b \in A.
%     \)
%     \\
%     Equivalently,
%     \(
%     \rho = \rho^{-1}.
%     \)
%     \\
%     Interpretations:
%     \begin{itemize}[noitemsep, topsep=-3pt]
%         \item In matrix form: the matrix of $\rho$ is symmetric.
%         \item In graphs: the graph is undirected (possibly with loops).
%     \end{itemize}
% }
% \tsDef{3.16 (Antisymmetric Relation)}{
%     A relation $\rho$ on a set $A$ is called \emph{antisymmetric} if
%     \(
%     (a \,\rho\, b \;\wedge\; b \,\rho\, a) \;\Rightarrow\; a = b
%     \ \forall a,b \in A.
%     \)
%     \\
%     Equivalently,
%     \(
%     \rho \cap \rho^{-1} \subseteq \mathrm{id}_A.
%     \)
% }
% \newline
% \tsDef{3.17 (Transitive Relation)}{
%     A relation $\rho$ on a set $A$ is called \emph{transitive} if
%     \(
%     (a \,\rho\, b \;\wedge\; b \,\rho\, c) \;\Longrightarrow\; a \,\rho\, c.
%     \)
% }
% \newline
% \newline
\tsLem{3.9}{
    A relation $\rho$ is transitive if and only if
    \(
    \rho^2 \subseteq \rho,
    \)
    where $\rho^2 = \rho \circ \rho$.
}
\newline
\tsDef{3.18 (Transitive Closure)}{
    The \emph{transitive closure} of a relation $\rho$ on a set $A$, denoted $\rho^{*}$, is defined by
    \(
    \rho^{*} := \bigcup_{n \in \mathbb{N}_{>0}} \rho^{n}.
    \)
}
\newline
\subsection{Equivalence Relation}
\tsDef{3.19 (Equivalence Relation)}{
    An equivalence relation on a set $A$ is a relation that is
    \emph{reflexive}, \emph{symmetric}, and \emph{transitive}.
}
\newline
\tsDef{3.20 (Equivalence Class)}{
    Let $\theta$ be an equivalence relation on a set $A$, and let $a \in A$.
    The \emph{equivalence class} of $a$ is
    \(
    [a]_{\theta} := \{\, b \in A \mid b \,\theta\, a \,\}.
    \)
    \\
    \textbf{Example:} (congruence modulo $3$ on $\mathbb{Z}$):
    \(
    [0] = \{\dots,-6,-3,0,3,6,\dots\},
    \)
    \(
    [1] = \{\dots,-5,-2,1,4,7,\dots\},
    \)
    \(
    [2] = \{\dots,-4,-1,2,5,8,\dots\}.
    \)
}
\newline
\tsLem{3.10}{
    The intersection of two equivalence relations on the same set is an equivalence relation.
}
\newline
\tsDef{3.21 (Partition)}{
    A \emph{partition} of a set $A$ is a family $\{S_i \mid i \in I\}$ of subsets of $A$
    such that
    \(
    S_i \cap S_j = \varnothing \ (i \neq j),
    \
    \bigcup_{i \in I} S_i = A.
    \)
}
\newline
\tsDef{3.22 (Quotient Set)}{
    Let $\theta$ be an equivalence relation on a set $A$.
    The set of equivalence classes is denoted by
    \(
    A / \theta := \{ [a]_{\theta} \mid a \in A \},
    \)
    and is called the \emph{quotient set} of $A$ modulo $\theta$.
}
\newline
\tsThe{3.11}{
    Let $\theta$ be an equivalence relation on a set $A$.
    Then the set $A/\theta$ of equivalence classes forms a partition of $A$.
}
\newline
\subsection{Partial Order Relations}
\tsDef{3.23 (Partial Order)}{
    A \emph{partial order} on a set $A$ is a relation that is
    \emph{reflexive}, \emph{antisymmetric}, and \emph{transitive}.
    \\
    A set equipped with a partial order $\preceq$ is called a
    \emph{partially ordered set} (poset), denoted $(A,\preceq)$.
    \\
    \textbf{Examples:}
    \(
    (\mathcal{P}(A), \subseteq) \text{ is a poset},
    \)
    \(
    (\mathbb{N}_{\ge 0}, \mid) \text{ is a poset},
    \)
    \(
    (\mathbb{Z}, \preceq) \text{ is a poset}.
    \)
    \\
    Note:
    \(
    a \prec b \;\Longleftrightarrow\; a \preceq b \;\wedge\; a \ne b.
    \)
}
\newline
\tsDef{3.24 (Comparable Elements)}{
    In a poset $(A,\preceq)$, two elements $a,b \in A$ are called \emph{comparable} if
    \(
    a \preceq b \ \text{or} \ b \preceq a.
    \)
    Otherwise, they are called \emph{incomparable}.
}
\newline
\tsDef{3.25 (Total Order)}{
    Let $(A,\preceq)$ be a poset.
    If any two elements of $A$ are comparable, then $A$ is called a
    \emph{totally ordered set} (or \emph{linearly ordered}) by $\preceq$.
    \\
    \textbf{Examples:}
    \(
    (\mathbb{Z},\le) \ \text{and} \ (\mathbb{Z},\ge)
    \)
    are totally ordered,
    \((\mathcal{P}(A),\subseteq)\) is not totally ordered if $|A|\ge 2$,
    \(
    (\mathbb{N},\mid)
    \) is not totally ordered
}
\newline
\tsDef{3.26 (Covering Relation)}{
    In a poset $(A,\preceq)$, an element $b$ \emph{covers} $a$ if: \\
    \(
    a \prec b \ \text{and there is no } c \text{ with } a \prec c \prec b
    \)
    between $a$ and $b$.
}
\newline
\tsDef{3.27 (Hasse Diagram)}{
    The \emph{Hasse diagram} of a finite poset $(A,\le)$ is the directed graph
    whose vertices are the elements of $A$, and where there is an edge from
    $a$ to $b$ if and only if $b$ covers $a$.
    \\
    \textbf{Example:}
    \(
    (\mathcal{P}(\{a,b,c\}),\subseteq).
    \)
}
\newline
\tsDef{3.28 (Product Order)}{
    Let $(A,\preceq_A)$ and $(B,\preceq_B)$ be posets.
    The product poset $(A\times B,\le)$ is defined by
    \(
    (a_1,b_1)\le (a_2,b_2)
    \;\Longleftrightarrow\;
    a_1 \preceq_A a_2 \;\wedge\; b_1 \preceq_B b_2.
    \)
}
\newline
\tsThe{3.12}{
    If $(A,\preceq_A)$ and $(B,\preceq_B)$ are posets, then
    \(
    (A,\preceq_A)\times(B,\preceq_B)
    \)
    is a partially ordered set.
}
\newline
\tsThe{3.13}{
    For posets $(A,\preceq_A)$ and $(B,\preceq_B)$, the relation
    \(
    (a_1,b_1)\le_{\text{lex}}(a_2,b_2)
    \;\Longleftrightarrow\;
    a_1\prec a_2 \;\vee\; (a_1=a_2 \wedge b_1\preceq_B b_2)
    \)
    defines a partial order on $A\times B$.
}
\newline
\tsDef{3.29 (Bounds)}{
    Let $(A,\preceq)$ be a poset and $S\subseteq A$. For $a\in A$:
    $\tsPoint$ $a$ is \emph{minimal / maximal} if there is no $b\in A$
    with $b\prec a$ / $b\succ a$.
    $\tsPoint$ $a$ is the \emph{least / greatest element} of $A$ if
    \(
    a\preceq b \ / \ a\succeq b \ \forall b\in A.
    \)
    $\tsPoint$ $a$ is a \emph{lower / upper bound} of $S$ if
    \(
    a\preceq b \ / \ a\succeq b \ \forall b\in S.
    \)
    $\tsPoint$ $a$ is the \emph{greatest lower bound} /
    \emph{least upper bound} of $S$ if it is respectively the
    greatest / least among all lower / upper bounds of $S$.
}
\newline
\tsDef{3.30 (Well-Ordered Set)}{
    A poset $(A,\preceq)$ is \emph{well-ordered} if it is totally ordered and
    every nonempty subset of $A$ has a least element.
    \(\tsPoint\) Every subset of a well-ordered set is also well-ordered.
}
\newline
\tsDef{3.31 (Meet and Join)}{
    Let $(A,\preceq)$ be a poset and $a,b\in A$.
    \(\tsPoint\) If $a$ and $b$ have a greatest lower bound, it is called the
    \emph{meet} and denoted $a\wedge b$.
    \(\tsPoint\) If $a$ and $b$ have a least upper bound, it is called the
    \emph{join} and denoted $a\vee b$.
}
\newline
\tsDef{3.32 (Lattice)}{
    A poset $(A,\preceq)$ in which every pair of elements has both a meet and a
    join is called a \emph{lattice}.
}
\subsection{Functions}
\tsDef{3.33 (Function)}{
    A function $f:A\to B$ from domain $A$ to codomain $B$ is a relation from
    $A$ to $B$ such that:
    \(\tsPoint\) For every $a\in A$ there exists $b\in B$ with $a\,f\,b$
    (totality).
    \(\tsPoint\) For all $a\in A$ and $b,b'\in B$,
    \(
    a\,f\,b \wedge a\,f\,b' \;\Longrightarrow\; b=b'
    \)
    (well-definedness).
    We write $f(a)=b$.
}
\newline
\tsDef{3.34}{
    The set of all functions from $A$ to $B$ is denoted $B^A$.
}
\newline
\tsDef{3.35 (Partial Function)}{
    A \emph{partial function} satisfies only condition (2) of Definition~3.33.
}
\newline
\tsDef{3.36 (Image of a Set)}{
    Let $f:A\to B$ be a function and $S\subseteq A$.
    The image of $S$ under $f$ is
    \(
    f(S):=\{f(a)\mid a\in S\}.
    \)
}
\newline
\tsDef{3.37 (Image of a Function)}{
    The image of $f$ is
    \(
    \operatorname{Im}(f):=f(A).
    \)
}
\newline
\tsDef{3.38 (Preimage)}{
    For $T\subseteq B$, the preimage of $T$ under $f$ is
    \(
    f^{-1}(T):=\{a\in A\mid f(a)\in T\}.
    \)
    \\
    \textbf{Example:}
    If $f(x)=x^2$, then
    \(
    f^{-1}(\{4,9\})=\{-3,-2,2,3\}.
    \)
}
\newline
\tsDef{3.39 (Injective, Surjective, Bijective)}{
    A function $f:A\to B$ is:
    \(\tsPoint\) \emph{Injective}: if $f(a)=f(a')\Rightarrow a=a'$.
    \(\tsPoint\) \emph{Surjective}: if $f(A)=B$.
    \(\tsPoint\) \emph{Bijective}: if it is both injective and surjective.
    \\
    A bijection has an inverse function $f^{-1}$.
}
\newline
\tsDef{3.41 (Composition of Functions)}{
    Let $f:A\to B$ and $g:B\to C$ be functions.
    \\
    Composition $g\circ f:A\to C$ is defined by
    \(
    (g\circ f)(a)=g(f(a)).
    \)
}
\newline
\tsLem{3.14}{
    Function composition is associative:
    \(
    (h\circ g)\circ f = h\circ(g\circ f).
    \)
}
\newline
\tsDef{3.42}{
    Two sets $A$ and $B$ are called \emph{equinumerous}, denoted $A \sim B$, if there exists a bijection
    \(
    f : A \to B.
    \)
    Equivalently, $A \sim B$ \(\Leftrightarrow\) $|A| = |B|$.
    \(\tsPoint\) A set $B$ \emph{dominates} a set $A$, denoted $A \preceq B$, if $A\sim C$ to some subset $C \subseteq B$.
    Equivalently, there exists an injective function $f : A \to B$.
    \(\tsPoint\) A set $A$ is called \emph{countable} if $A \preceq \mathbb{N}$, and \emph{uncountable} otherwise.
    Equivalently, $A$ is countable if there exists an injection $f : A \to \mathbb{N}$.
}
\newline
\tsLem{3.15}{
    The relation $\sim$ is an equivalence relation.
    \(\tsPoint\) The relation $\preceq$ is transitive:
    \(
    A \preceq B \;\wedge\; B \preceq C \;\Rightarrow\; A \preceq C.
    \)
    \(\tsPoint\) If $A \subseteq B$, then $A \preceq B$.
}
\newline
\tsThe{3.16 (Bernstein-Schröder theorem)}{
    If $A \preceq B$ and $B \preceq A$, then $A \sim B$.
}
\newline
\tsThe{3.17}{
    A set $A$ is countable if and only if $A$ is finite or $A \sim \mathbb{N}$.
}
\newline
\tsThe{3.18}{
    The set $\{0,1\}^*$ of all finite binary sequences is countable.
}
\newline
\tsThe{3.19}{
    The set $\mathbb{N} \times \mathbb{N}$ of ordered pairs of natural numbers is countable.
}
\newline
\tsCor{3.20}{
    If $A$ and $B$ are countable sets, then their Cartesian product $A \times B$ is countable:
    \(A \preceq \mathbb{N} \land B \preceq \mathbb{N} \Longrightarrow A \times B \preceq \mathbb{N}\)
}
\newline
\tsCor{3.21}{
    The set of rational numbers $\mathbb{Q}$ is countable.
    \\
    \textbf{Idea:}
    Every rational number can be written as $\frac{m}{n}$ with $m \in \mathbb{Z}$ and $n \in \mathbb{N}_{>0}$.
    Thus $\mathbb{Q} \preceq \mathbb{Z} \times \mathbb{N}$, which is countable.
}
\newline
\tsThe{3.22}{
    Let $A$ and $\{A_i\}_{i \in \mathbb{N}}$ be countable sets, then:
    \(\tsPoint\) For any $n \in \mathbb{N}$, the Cartesian product $A^n$ is countable.
    \(\tsPoint\) The union $\bigcup_{i \in \mathbb{N}} A_i$ is countable.
    \(\tsPoint\) The set $A^*$ of all finite sequences with elements from $A$ is countable.
}
\tsDef{3.23}{
    Let $\{0,1\}^{\mathbb{N}}$ denote the set of infinite binary sequences, equivalently the set of functions
    \(
    f : \mathbb{N} \to \{0,1\}.
    \)
}
\newline
\tsThe{3.23}{
    The set $\{0,1\}^{\mathbb{N}}$ is uncountable.
    \\
    \textbf{Idea:} This follows from \textit{Cantor’s diagonal argument}.
}
\newline
\tsDef{3.44}{
    A function
    \(
    f : \mathbb{N} \to \{0,1\}
    \)
    is called \emph{computable} if there exists a program such that, for every $n \in \mathbb{N}$, the program outputs $f(n)$ when given input $n$.
}
\newline
\tsCor{3.24}{
    There exist uncomputable functions $f : \mathbb{N} \to \{0,1\}$.
    \\
    \textbf{Remark.}
    The Halting Problem gives an explicit example of an uncomputable function.
}