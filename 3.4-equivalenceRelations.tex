\subsection{Equivalence Relation}
\tsDef{3.19 (Equivalence Relation)}{
    An equivalence relation on a set $A$ is a relation that is
    \emph{reflexive}, \emph{symmetric}, and \emph{transitive}.
}
\newline
\tsDef{3.20 (Equivalence Class)}{
    Let $\theta$ be an equivalence relation on a set $A$, and let $a \in A$.
    The \emph{equivalence class} of $a$ is
    \(
    [a]_{\theta} := \{\, b \in A \mid b \,\theta\, a \,\}.
    \)
    \\
    \textbf{Example:} (congruence modulo $3$ on $\mathbb{Z}$):
    \(
    [0] = \{\dots,-6,-3,0,3,6,\dots\},
    \)
    \(
    [1] = \{\dots,-5,-2,1,4,7,\dots\},
    \)
    \(
    [2] = \{\dots,-4,-1,2,5,8,\dots\}.
    \)
}
\newline
\tsLem{3.10 (Intersection of equivalence relaltions)}{
    The intersection of two equivalence relations on the same set is an equivalence relation.
}
\newline
\tsDef{3.21 (Partition)}{
    A \emph{partition} of a set $A$ is a family $\{S_i \mid i \in I\}$ of subsets of $A$
    such that
    \(
    S_i \cap S_j = \varnothing \ (i \neq j),
    \
    \bigcup_{i \in I} S_i = A.
    \)
}
\newline
\tsDef{3.22 (Quotient Set)}{
    Let $\theta$ be an equivalence relation on a set $A$.
    The set of equivalence classes is denoted by
    \(
    A / \theta := \{ [a]_{\theta} \mid a \in A \},
    \)
    and is called the \emph{quotient set} of $A$ modulo $\theta$.
}
\newline
\tsThe{3.11 (Set of equiv. classes forms a partition of a set)}{
    Let $\theta$ be an equivalence relation on a set $A$.
    Then the set $A/\theta$ of equivalence classes forms a partition of $A$.
}
\newline
