\subsection*{{Extended Euclidean Algorithm}}
Each remainder in the Euclidean algorithm is a linear combination of the
initial integers. The last nonzero remainder is the gcd.
\[
    \begin{array}{rcl}
        252 & = & 1\cdot198 + 54 \\
        198 & = & 3\cdot54  + 36 \\
        54  & = & 1\cdot36  + 18 \\
        36  & = & 2\cdot18
    \end{array}
    \qquad
    \Rightarrow
    \qquad
    18 = 4\cdot252 - 5\cdot198.
\]
\hrule
\subsection*{Computing big exponents}
Compute $a^N \pmod{p}.$
\(\tsPoint\) Find a small $k \text{ such that } a^k \equiv r \pmod{p}$.
\(\tsPoint\) Write $N = kq \;(+\; r_0).$
\[
    a^N = (a^k)^q \cdot a^{r_0}
    \equiv r^q \cdot a^{r_0} \pmod{p}.
\]
Evaluate and conclude. Example:
\\
\(
2^6 \equiv -1 \pmod{13}, \quad 4536 = 6\cdot756
\Rightarrow 2^{4536} \equiv (-1)^{756} \equiv 1 \pmod{13}.
\)
\newline
\hrule
\subsection*{Chinese Remainder Theorem}
Solve the system
\[
    x \equiv_3 2, \quad x \equiv_5 3, x \equiv_7 2
\]

\textbf{Step 1: Combine moduli.}
\[
    N = 3 \cdot 5 \cdot 7 = 105,
    \qquad
    N_1 = \frac{N}{3} = 35,\;
    N_2 = \frac{N}{5} = 21,\;
    N_3 = \frac{N}{7} = 15.
\]

\textbf{Step 2: Compute inverses.}
\[
    35^{-1} \equiv_3 2,\qquad
    21^{-1} \equiv_5 1,\qquad
    15^{-1} \equiv_7 1.
\]

\textbf{Step 3: Assemble the solution.}
\[
    x \equiv \sum a_i N_i N_i^{-1} \pmod{N},
\]
\[
    x \equiv
    2 \cdot 35 \cdot 2
    + 3 \cdot 21 \cdot 1
    + 2 \cdot 15 \cdot 1
    \pmod{105}.
\]
\[
    x \equiv 233 \equiv \boxed{23 \pmod{105}}.
\]
\hrule
\subsection*{Polynomial Interpolation}
\[
    a(x) = \sum_{i=0}^{n} y_i \, L_i(x),
\]
where
\(
L_i(x) = \prod_{\substack{j=0 \\ j \neq i}}^{n}
\frac{x - x_j}{x_i - x_j}.
\)
\\ \\
Given the data points:
\(
(0,1),\ (1,3),\ (2,2),
\)
we obtain:
\[
    L_0(x) = \frac{(x-1)(x-2)}{(0-1)(0-2)}, \qquad
    L_1(x) = \frac{(x-0)(x-2)}{(1-0)(1-2)}, \qquad
    L_2(x) = \frac{(x-0)(x-1)}{(2-0)(2-1)}.
\]
Therefore,
\[
    a(x) = 1 \cdot L_0(x) + 3 \cdot L_1(x) + 2 \cdot L_2(x).
\]
\subsection*{Diffie-Hellman Key-Agreement}
\begin{itemize}[noitemsep, topsep=-2px]
    \item Alice and Bob select a random $x_A, x_B \in \{0,\dots,p-2\}$.
    \item Alice computes $y_A = R_p(g^{x_A})$, Bob computes $y_B = R_p(g^{x_B})$.
    \item They exchange $y_A$ and $y_B$.
    \item Alice computes $k_{AB} = R_p(y_B^{x_A})$, Bob computes
          $k_{BA} = R_p(y_A^{x_B})$.
\end{itemize}
Then
\[
    k_{AB}
    \equiv y_B^{x_A}
    \equiv (g^{x_B})^{x_A}
    \equiv g^{x_A x_B}
    \equiv k_{BA}
    \pmod{p}.
\]
\hrule
\subsection*{RSA Public-Key Encryption}
Define a group $G$ and choose two large primes $p,q$.
\begin{itemize}[noitemsep, topsep=-1px]
    \item $n = pq$
    \item $|G| = |\mathbb{Z}_n^*| = |\mathbb{Z}_{pq}^*|
              = \varphi(n) = (p-1)(q-1)$
\end{itemize}
Let $e \in \mathbb{Z}$ be relatively prime to $|G|$ and let
\(
d \equiv e^{-1} \pmod{|G|}.
\)
\\
Then the map
\(
x \mapsto x^e
\)
is a bijection on $G$, and for all ciphertexts $c = x^e$ we have
\(
x = c^d = x^{ed}.
\)
\\
\textbf{Proof.}
Since $ed = k|G| + 1$ for some $k \in \mathbb{Z}$,
\[
    x^{ed} = x^{k|G|+1} = (x^{|G|})^k x = x.
\]

\subsubsection*{Application}
\begin{itemize}[noitemsep, topsep=-1px]
    \item Select $e$ and compute $d \equiv e^{-1} \pmod{|G|}$
    \item Publish the public key $(n,e)$
    \item The other party computes $c = R_n(m^e)$ and sends $c$
    \item You recover the message by computing $m = R_n(c^d)$
\end{itemize}
\textbf{RSA example (small primes).}
\\
Choose $p=5$, $q=11$, so $n=pq=55$ and $\varphi(n)=(p-1)(q-1)=40$.
\\
Choose $e=3$ with $\gcd(3,40)=1$, and compute
$d\equiv e^{-1}\pmod{40}\equiv27$.
\\
Public key $(n,e)=(55,3)$, private key $d=27$.
\\
For message $m=7$, encrypt
$c\equiv m^e\equiv7^3\equiv13\pmod{55}$,
\\
and decrypt
$m\equiv c^d\equiv13^{27}\equiv7\pmod{55}$.


% \hrule
% \subsection*{Error-Correcting Codes}