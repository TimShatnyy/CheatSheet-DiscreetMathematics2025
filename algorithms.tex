\subsection*{{Extended Euclidean Algorithm}}
Each remainder in the Euclidean algorithm is a linear combination of the
initial integers. The last nonzero remainder is the gcd.
\[
    \begin{array}{rcl}
        252 & = & 1\cdot198 + 54 \\
        198 & = & 3\cdot54  + 36 \\
        54  & = & 1\cdot36  + 18 \\
        36  & = & 2\cdot18
    \end{array}
    \qquad
    \Rightarrow
    \qquad
    18 = 4\cdot252 - 5\cdot198.
\]
\hrule
\subsection*{Computing big exponents}
Compute $a^N \pmod{p}.$
\(\tsPoint\) Find a small $k \text{ such that } a^k \equiv r \pmod{p}$.
\(\tsPoint\) Write $N = kq \;(+\; r_0).$
\[
    a^N = (a^k)^q \cdot a^{r_0}
    \equiv r^q \cdot a^{r_0} \pmod{p}.
\]
Evaluate and conclude. Example:
\\
\(
2^6 \equiv -1 \pmod{13}, \quad 4536 = 6\cdot756
\Rightarrow 2^{4536} \equiv (-1)^{756} \equiv 1 \pmod{13}.
\)
\newline
\hrule
\subsection*{Chinese Remainder Theorem}
Solve the system
\[
    x \equiv_3 2, \quad x \equiv_5 3, x \equiv_7 2
\]

\textbf{Step 1: Combine moduli.}
\[
    N = 3 \cdot 5 \cdot 7 = 105,
    \qquad
    N_1 = \frac{N}{3} = 35,\;
    N_2 = \frac{N}{5} = 21,\;
    N_3 = \frac{N}{7} = 15.
\]

\textbf{Step 2: Compute inverses.}
\[
    35^{-1} \equiv_3 2,\qquad
    21^{-1} \equiv_5 1,\qquad
    15^{-1} \equiv_7 1.
\]

\textbf{Step 3: Assemble the solution.}
\[
    x \equiv \sum a_i N_i N_i^{-1} \pmod{N},
\]
\[
    x \equiv
    2 \cdot 35 \cdot 2
    + 3 \cdot 21 \cdot 1
    + 2 \cdot 15 \cdot 1
    \pmod{105}.
\]
\[
    x \equiv 233 \equiv \boxed{23 \pmod{105}}.
\]
\hrule
\subsection*{Polynomial Interpolation}
\[
    a(x) = \sum_{i=0}^{n} y_i \, L_i(x),
\]
where
\(
L_i(x) = \prod_{\substack{j=0 \\ j \neq i}}^{n}
\frac{x - x_j}{x_i - x_j}.
\)
\\ \\
Given the data points:
\(
(0,1),\ (1,3),\ (2,2),
\)
we obtain:
\[
    L_0(x) = \frac{(x-1)(x-2)}{(0-1)(0-2)}, \
    L_1(x) = \frac{(x-0)(x-2)}{(1-0)(1-2)}, \
    L_2(x) = \frac{(x-0)(x-1)}{(2-0)(2-1)}
\]
Therefore,
\[
    a(x) = 1 \cdot L_0(x) + 3 \cdot L_1(x) + 2 \cdot L_2(x).
\]
\hrule
\subsection*{Diffie-Hellman Key-Agreement}
\begin{itemize}[noitemsep, topsep=-2px]
    \item Alice and Bob select a random $x_A, x_B \in \{0,\dots,p-2\}$.
    \item Alice computes $y_A = R_p(g^{x_A})$, Bob computes $y_B = R_p(g^{x_B})$.
    \item They exchange $y_A$ and $y_B$.
    \item Alice computes $k_{AB} = R_p(y_B^{x_A})$, Bob computes
          $k_{BA} = R_p(y_A^{x_B})$.
\end{itemize}
Then
\[
    k_{AB}
    \equiv y_B^{x_A}
    \equiv (g^{x_B})^{x_A}
    \equiv g^{x_A x_B}
    \equiv k_{BA}
    \pmod{p}.
\]
% \hrule
\subsection*{RSA Public-Key Encryption}
Define a group $G$ and choose two large primes $p,q$.
\begin{itemize}[noitemsep, topsep=-1px]
    \item $n = pq$
    \item $|G| = |\mathbb{Z}_n^*| = |\mathbb{Z}_{pq}^*|
              = \varphi(n) = (p-1)(q-1)$
\end{itemize}
Let $e \in \mathbb{Z}$ be relatively prime to $|G|$ and let
\(
d \equiv e^{-1} \pmod{|G|}.
\)
\\
Then the map
\(
x \mapsto x^e
\)
is a bijection on $G$, and for all ciphertexts $c = x^e$ we have
\(
x = c^d = x^{ed}.
\)
\\
\textbf{Proof.}
Since $ed = k|G| + 1$ for some $k \in \mathbb{Z}$,
\[
    x^{ed} = x^{k|G|+1} = (x^{|G|})^k x = x.
\]

\subsubsection*{Application}
\begin{itemize}[noitemsep, topsep=-1px]
    \item Select $e$ and compute $d \equiv e^{-1} \pmod{|G|}$
    \item Publish the public key $(n,e)$
    \item The other party computes $c = R_n(m^e)$ and sends $c$
    \item You recover the message by computing $m = R_n(c^d)$
\end{itemize}
\textbf{RSA example (small primes).}
\\
Choose $p=5$, $q=11$, so $n=pq=55$ and $\varphi(n)=(p-1)(q-1)=40$.
\\
Choose $e=3$ with $\gcd(3,40)=1$, and compute
$d\equiv e^{-1}\pmod{40}\equiv27$.
\\
Public key $(n,e)=(55,3)$, private key $d=27$.
\\
For message $m=7$, encrypt
$c\equiv m^e\equiv7^3\equiv13\pmod{55}$,
\\
and decrypt
$m\equiv c^d\equiv13^{27}\equiv7\pmod{55}$.
\newline
\hrule
\subsection*{Primes:}
Primes: 2, 3, 5, 7, 11, 13, 17, 19, 23, 29, 31, 37, 41, 43, 47, 53, 59, 61, 67, 71, 73, 79, 83, 89,
97, 101, 103, 107, 109, 113, 127, 131, 137, 139, 149, 151, 157, 163, 167, 173 , 179, 181 ...
% \subsection*{Error-Correcting Codes}
\vspace{10cm}
\subsection*{Modular Inverses:}
\textit{Modular inverses: entry $(m,a)$ equals $a^{-1}\pmod m$.}
\begin{center}
    \small
    \setlength{\tabcolsep}{0.9pt}
    \renewcommand{\arraystretch}{1}
    \begin{tabular}{|c|*{25}{c|}}
        \hline
        $m\backslash a$
           & 1 & 2  & 3  & 4  & 5  & 6  & 7  & 8  & 9  & 10 & 11 & 12 & 13 & 14 & 15 & 16 & 17 & 18 & 19 & 20 & 21 & 22 & 23 & 24 & 25 \\  \hline
        1  & 0 & 0  & 0  & 0  & 0  & 0  & 0  & 0  & 0  & 0  & 0  & 0  & 0  & 0  & 0  & 0  & 0  & 0  & 0  & 0  & 0  & 0  & 0  & 0  & 0  \\ \hline
        2  & 1 &    & 1  &    & 1  &    & 1  &    & 1  &    & 1  &    & 1  &    & 1  &    & 1  &    & 1  &    & 1  &    & 1  &    & 1  \\ \hline
        3  & 1 & 2  &    & 1  & 2  &    & 1  & 2  &    & 1  & 2  &    & 1  & 2  &    & 1  & 2  &    & 1  & 2  &    & 1  & 2  &    & 1  \\ \hline
        4  & 1 &    & 3  &    & 1  &    & 3  &    & 1  &    & 3  &    & 1  &    & 3  &    & 1  &    & 3  &    & 1  &    & 3  &    & 1  \\ \hline
        5  & 1 & 3  & 2  & 4  &    & 1  & 3  & 2  & 4  &    & 1  & 3  & 2  & 4  &    & 1  & 3  & 2  & 4  &    & 1  & 3  & 2  & 4  &    \\ \hline
        6  & 1 &    &    &    & 5  &    & 1  &    &    &    & 5  &    & 1  &    &    &    & 5  &    & 1  &    &    &    & 5  &    & 1  \\ \hline
        7  & 1 & 4  & 5  & 2  & 3  & 6  &    & 1  & 4  & 5  & 2  & 3  & 6  &    & 1  & 4  & 5  & 2  & 3  & 6  &    & 1  & 4  & 5  & 2  \\ \hline
        8  & 1 &    & 3  &    & 5  &    & 7  &    & 1  &    & 3  &    & 5  &    & 7  &    & 1  &    & 3  &    & 5  &    & 7  &    & 1  \\ \hline
        9  & 1 & 5  &    & 7  & 2  &    & 4  & 8  &    & 1  & 5  &    & 7  & 2  &    & 4  & 8  &    & 1  & 5  &    & 7  & 2  &    & 4  \\ \hline
        10 & 1 &    & 7  &    &    &    & 3  &    & 9  &    & 1  &    & 7  &    &    &    & 3  &    & 9  &    & 1  &    & 7  &    &    \\ \hline
        11 & 1 & 6  & 4  & 3  & 9  & 2  & 8  & 7  & 5  & 10 &    & 1  & 6  & 4  & 3  & 9  & 2  & 8  & 7  & 5  & 10 &    & 1  & 6  & 4  \\ \hline
        12 & 1 &    &    &    & 5  &    & 7  &    &    &    & 11 &    & 1  &    &    &    & 5  &    & 7  &    &    &    & 11 &    & 1  \\ \hline
        13 & 1 & 7  & 9  & 10 & 8  & 11 & 2  & 5  & 3  & 4  & 6  & 12 &    & 1  & 7  & 9  & 10 & 8  & 11 & 2  & 5  & 3  & 4  & 6  & 12 \\ \hline
        14 & 1 &    & 5  &    & 3  &    &    &    & 11 &    & 9  &    & 13 &    & 1  &    & 5  &    & 3  &    &    &    & 11 &    & 9  \\ \hline
        15 & 1 & 8  &    & 4  &    &    & 13 & 2  &    &    & 11 &    & 7  & 14 &    & 1  & 8  &    & 4  &    &    & 13 & 2  &    &    \\ \hline
        16 & 1 &    & 11 &    & 13 &    & 7  &    & 9  &    & 3  &    & 5  &    & 15 &    & 1  &    & 11 &    & 13 &    & 7  &    & 9  \\ \hline
        17 & 1 & 9  & 6  & 13 & 7  & 3  & 5  & 15 & 2  & 12 & 14 & 10 & 4  & 11 & 8  & 16 &    & 1  & 9  & 6  & 13 & 7  & 3  & 5  & 15 \\ \hline
        18 & 1 &    &    &    & 11 &    & 13 &    &    &    & 5  &    & 7  &    &    &    & 17 &    & 1  &    &    &    & 11 &    & 13 \\ \hline
        19 & 1 & 10 & 13 & 5  & 4  & 16 & 11 & 12 & 17 & 2  & 7  & 8  & 3  & 15 & 14 & 6  & 9  & 18 &    & 1  & 10 & 13 & 5  & 4  & 16 \\ \hline
        20 & 1 &    & 7  &    &    &    & 3  &    & 9  &    & 11 &    & 17 &    &    &    & 13 &    & 19 &    & 1  &    & 7  &    &    \\ \hline
        21 & 1 & 11 &    & 16 & 17 &    &    & 8  &    & 19 & 2  &    & 13 &    &    & 4  & 5  &    & 10 & 20 &    & 1  & 11 &    & 16 \\ \hline
        22 & 1 &    & 15 &    & 9  &    & 19 &    & 5  &    &    &    & 17 &    & 3  &    & 13 &    & 7  &    & 21 &    & 1  &    & 15 \\ \hline
        23 & 1 & 12 & 8  & 6  & 14 & 4  & 10 & 3  & 18 & 7  & 21 & 2  & 16 & 5  & 20 & 13 & 19 & 9  & 17 & 15 & 11 & 22 &    & 1  & 12 \\  \hline
        24 & 1 &    &    &    & 5  &    & 7  &    &    &    & 11 &    & 13 &    &    &    & 17 &    & 19 &    &    &    & 23 &    & 1  \\  \hline
        25 & 1 & 13 & 17 & 19 &    & 21 & 18 & 22 & 14 &    & 16 & 23 & 2  & 9  &    & 11 & 3  & 7  & 4  &    & 6  & 8  & 12 & 24 &    \\ \hline
    \end{tabular}
\end{center}
