\subsection{The Structure of Groups}
\tsDef{5.10 (Direct Product of Groups)}{
    The direct product of groups
    \(
    \langle G_1, *_1 \rangle, \dots, \langle G_n, *_n \rangle
    \)
    is the algebra
    \(
    \langle G_1 \times \cdots \times G_n, * \rangle,
    \)
    where
    \(
    (a_1, \dots, a_n) * (b_1, \dots, b_n)
    =
    (a_1 *_1 b_1, \dots, a_n *_n b_n).
    \)
}
\newline
\tsDef{5.11 (Group Homomorphism)}{
    For two groups $\langle G, *, \widehat{}, e \rangle$ and $\langle H, \star, \widetilde{\ }, e' \rangle$, a function
    \(
    \psi : G \to H
    \)
    is a \emph{group homomorphism} if
    \(
    \psi(a * b) = \psi(a) \star \psi(b) \ \forall a,b \in G.
    \)
    \\
    If $\psi$ is bijective, it is called an \emph{isomorphism}, and we write
    \(
    G \simeq H.
    \)
}
\newline
\tsLem{5.12 (Homomorphism properties)}{
    A group homomorphism $\psi : G \to H$ satisfies:
    \begin{enumerate}[noitemsep, topsep=-1px]
        \item $\psi(e) = e'$,
        \item $\psi(\widehat{a}) = \widetilde{(\psi(a))}$ for all $a \in G$.
    \end{enumerate}
}
\tsDef{5.13 (Subgroup)}{
    A subset $H \subseteq G$ of a group $\langle G, *, \widehat{\ }, e \rangle$ is called a \emph{subgroup} if $\langle H, *, \widehat{\ }, e \rangle$ is itself a group, i.e.:
    \begin{enumerate}[noitemsep, topsep=-1px]
        \item $a * b \in H$ for all $a,b \in H$,
        \item $e \in H$,
        \item $a^{-1} \in H$ for all $a \in H$.
    \end{enumerate}
}
\tsDef{5.14 (Order of an Element)}{
    Let $G$ be a group and $a \in G$. The \emph{order} of $a$, denoted $\mathrm{ord}(a)$, is the smallest $m \ge 1$ such that
    \(
    a^m = e,
    \)
    if such $m$ exists. Otherwise, $\mathrm{ord}(a) = \infty$.
}
\newline
\tsLem{5.15 (Finite groups - finite order of elements)}{
    In a finite group, every element has finite order.
}
\newline
\tsDef{5.16 (Order of a Group)}{
    For a finite group $G$, the number $|G|$ is called the \emph{order of the group}.
}
\newline
\tsDef{5.17 (Generated Subgroup)}{
    For a group $G$ and $a \in G$, the subgroup generated by $a$ is defined as
    \(
    \langle a \rangle := \{ a^n \mid n \in \mathbb{Z} \}.
    \)
    It is the smallest subgroup of $G$ containing $a$.
    \(\tsPoint\)
    If $G$ is finite, then
    \(
    \langle a \rangle = \{ e, a, a^2, \dots, a^{\mathrm{ord}(a)-1} \}.
    \)
}
\newline
\tsDef{5.15 (Cyclic Group)}{
    A group $G = \langle g \rangle$ generated by a single element $g \in G$ is called \emph{cyclic}, and $g$ is called a \emph{generator} of $G$.
    \\
    If $g$ is a generator, then so is $g^{-1}$.
    \\
    The generators of $\langle \mathbb{Z}_n, + \rangle$ are all $a \in \mathbb{Z}_n$ such that $\gcd(a,n)=1$.
    \\
    If a group is cyclic, then there exists an element $x$ such that every member of $G$ is a power of $x$.
}
\newline
\tsThe{5.7 (Classification of Cyclic Groups)}{
    A cyclic group of order $n$ is isomorphic to $\langle \mathbb{Z}_n, + \rangle$ and hence is abelian.
    $\langle \mathbb{Z}_n, + \rangle$ is the standard notation for a cyclic group of order $n$.
}
\newline
\tsThe{5.8 (Lagrange's Theorem)}{
    Let $G$ be a finite group and let $H$ be a subgroup of $G$. Then
    \(
    |H| \mid |G|.
    \)
}
\newline
\tsCor{5.9 (Division of order of fin. group by ord. of elements)}{
    For a finite group $G$, the order of every element divides the order of the group, i.e.
    \(
    \operatorname{ord}(a) \mid |G| \quad \text{for all } a \in G.
    \)
}
\newline
\tsCor{5.10 (Group order yields the identity)}{
    Let $G$ be a finite group. Then
    \(
    a^{|G|} = e \quad \text{for all } a \in G.
    \)
}
\newline
\tsThe{5.11 (Prime order groups are cyclic)}{
    Every group of prime order is cyclic, and in such a group every element except the neutral element is a generator.
}
\newline
\tsDef{5.16 (Multiplicative Group of Units)}{
    Let
    \(
    \mathbb{Z}_m^* := \{ a \in \mathbb{Z}_m \mid \gcd(a,m)=1 \}.
    \)
    Then $\mathbb{Z}_m^*$ forms a group under multiplication modulo $m$.
    It consists exactly of those elements that admit a multiplicative inverse modulo $m$. These elements are called the \emph{units} of $\mathbb{Z}_m$.
}
\newline
\tsDef{5.17 (Euler Totient Function)}{
    The Euler totient function $\varphi : \mathbb{Z}^+ \to \mathbb{Z}^+$ is defined by
    \(
    \varphi(m) := |\mathbb{Z}_m^*|.
    \)
}
\newline
\tsThe{5.12 (Totient Formula)}{
    If the prime factorization of $m$ is
    \(
    m = \prod_{i=1}^r p_i^{e_i},
    \)
    then \\
    \(
    \varphi(m) = \prod_{i=1}^r (p_i - 1)p_i^{e_i - 1}.
    \)
    \\
    Equivalently,
    \(
    \varphi(m) = m \prod_{p \mid m} \left(1 - \frac{1}{p}\right),
    \)
    where the product is over all primes dividing $m$.
}
\newline
\tsThe{5.13 (Multiplicative group from units)}{
    $\langle \mathbb{Z}_m^*, \cdot, 1 \rangle$ is a group.
}
% \newline
% \tsDef{5.14 (Example)}{
%     For $m=18$,
%     \(
%     \mathbb{Z}_{18}^* = \{1,5,7,11,13,17\},
%     \)
%     and we have $5^{-1} \equiv 11 \pmod{18}$ since $5 \cdot 11 \equiv 1 \pmod{18}$.
% }
\newline
\tsCor{5.14 (Fermat, Euler: Totient power gives the identity)}{
    $\forall m \ge 2$ and $\forall a$ such that $\gcd(a,m)=1$: \
    \(
    a^{\varphi(m)} \equiv_m 1.
    \)
    \\
    In particular, for every prime $p$ and every $a \not\equiv_p 0$: \
    \(
    a^{p-1} \equiv_p 1.
    \)
}
\newline
\tsThe{5.15 (Cyclicity criterion for $\mathbb{Z}_m^*$)}{
    The group $\mathbb{Z}_m^*$ is cyclic if and only if
    \(
    m = 2,\; 4,\; p^e,\; \text{or } 2p^e,
    \)
    where $p$ is an odd prime and $e \ge 1$.
}
\newline
\tsThe{5.16 (Coprime exponent bijection)}{
    If $G$ is finite and $\gcd(e,|G|)=1$, then:
    \\
    \(
    x \mapsto x^e \text{ is a bijection on } G,
    \
    x^e=y \iff x=y^d,
    \)
    where $d$ is the mult. inverse of $e$ modulo $|G|$:
    \(
    \ ed\equiv_{|G|} 1.
    \)
}
