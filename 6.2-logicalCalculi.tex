\subsection{Logical Calculi}
\tsDef{6.17 (Derivation Rule)}{
    Let $R$ be a rule. If $G$ can be obtained from $\{F_1,\dots,F_k\}$ using rule $R$, we write
    \(
    \{F_1,\dots,F_k\} \vdash_R G.
    \)
    Derivation is a purely syntactic concept.
}
\newline
\tsDef{6.18 (Application of derivation rules)}{
    The application of a derivation rule $R$ to a set $M$ of formulas means:
    \begin{enumerate}[noitemsep, topsep=-1pt]
        \item Select a subset $N \subseteq M$ such that $N \vdash_R G$ for some formula $G$.
        \item Add $G$ to $M$, i.e.\ replace $M$ by $M \cup \{G\}$.
    \end{enumerate}
}
\tsDef{6.19 (Calculus)}{
    A calculus $K$ is a finite set of derivation rules:
    \(
    K=\{R_1,\dots,R_n\}.
    \)
}
\newline
\tsDef{6.20 (Derivation)}{
    A derivation of $G$ from $M$ in calculus $K$ is a finite application of rules in $K$ leading to $G$.
    We write
    \(
    M \vdash_K G.
    \)
}
\newline
\tsDef{6.22 (Calculus soundness and completeness)}{
    A calculus $K$ is \emph{sound} if for all sets $M$ of formulas and all formulas $F$,
    \\
    \(
    M \vdash_K F \;\Rightarrow\; M \models F.
    \)
    \\
    \(\bullet\) It is \emph{complete} if for all $M$ and $F$,
    \(
    M \models F \;\Rightarrow\; M \vdash_K F.
    \)
}
