\subsection{Sets}
\tsDef{3.2 (Equal sets)}{
    For sets $A$ and $B$,
    \(
    A = B \;\Longleftrightarrow\; \forall x \,(x \in A \iff x \in B).
    \)
}
\newline
\tsLem{3.1 (Equality of set elemetns and ord. pairs)}{
    For any sets $A$ and $B$,
    \(
    \{A\} = \{B\} \;\Longrightarrow\; A = B.
    \)
    \\
    \textbf{Ordered pairs:}
    \(
    (a,b) = (c,d) \;\Longleftrightarrow\; a = c \;\wedge\; b = d.
    \)
    \\
    \textbf{Odered pairs via sets:}
    \(
    (a,b) := \bigl\{ \{a\}, \{a,b\} \bigr\}.
    \)
}
\newline
\tsDef{3.3 (Subset)}{
    \(
    A \subseteq B \;\Longleftrightarrow\; \forall x \,(x \in A \Rightarrow x \in B).
    \)
}
\newline
\tsLem{3.2 (Sets equality and subsets)}{
    \(
    A = B \;\Longleftrightarrow\; (A \subseteq B) \wedge (B \subseteq A).
    \)
    Equivalently: \\
    \(
    \forall x \,\bigl((x \in A \Rightarrow x \in B) \wedge (x \in B \Rightarrow x \in A)\bigr)
    \;\Leftrightarrow\;
    \forall x \,(x \in A \iff x \in B).
    \)
}
\newline
\tsLem{3.3 (Transitivity of subsets)}{
    For all sets $A,B,C$,
    \(
    A \subseteq B \;\wedge\; B \subseteq C \;\Longrightarrow\; A \subseteq C.
    \)
}
\newline
\tsDef{3.4 (Union and Intersection)}{
    \(
    A \cup B := \{x \mid x \in A \vee x \in B\},
    \)
    \(
    A \cap B := \{x \mid x \in A \wedge x \in B\}.
    \)
    \\
    \textbf{Families of Sets:}
    Let $\mathcal{A}$ be a set of sets:\\
    \(
    \bigcap \mathcal{A} := \{x \mid x \in A \text{ for all } A \in \mathcal{A}\},
    \)
    \(
    \bigcup \mathcal{A} := \{x \mid x \in A \text{ for some } A \in \mathcal{A}\}.
    \)
    % \\
    % \textbf{Example:}
    % \\
    % \(
    % \mathcal{A} = \bigl\{\{a,b,c\}, \{a,c\}, \{a,b,c,f\}, \{a,c,d\}\bigr\},
    % \)
    % \\
    % \(
    % \bigcup \mathcal{A} = \{a,b,c,d,e,f\},
    % \qquad
    % \bigcap \mathcal{A} = \{a,c\}.
    % \)
    \\
    If $I$ is an index set and $\mathcal{A} = \{A_i \mid i \in I\}$, then
    \(
    \bigcap_{i \in I} A_i,
    \
    \bigcup_{i \in I} A_i.
    \)
}
\newline
\tsDef{3.5 (Set Difference)}{
    The difference of sets $B$ and $A$ is
    \(
    B \setminus A := \{x \in B \mid x \notin A\}.
    \)
}
\newline
\tsDef{3.6 (Empty Set)}{
    A set is called \emph{empty} if it contains no elements:
    \(
    \forall x \,(x \notin A).
    \)
}
\newline
\tsLem{3.5 (Uniqueness of an empty set)}{
    There is exactly \textbf{one} empty set, denoted $\varnothing$ or $\{\}$.
}
\newline
\tsLem{3.6 (Empty set is a subset of every set)}{
    The empty set is a subset of every set:
    \(
    \forall A \;(\varnothing \subseteq A).
    \)
    \\
    \textbf{Construction of natural numbers:}
    \(
    S(n) := n \cup \{n\} \ \text{(rec. sucsessor)}.
    \)
}
\newline
\tsDef{3.7 (Power Set)}{
    The power set of a set $A$, denoted $\mathcal{P}(A)$, is the set of all subsets of $A$:
    \(
    \mathcal{P}(A) := \{S \mid S \subseteq A\}.
    \)
    If $|A| = k$, then
    \(
    |\mathcal{P}(A)| = 2^k.
    \)
    In particular, for a set with $k$ elements, each element may be
    \emph{included} or \emph{excluded}, giving
    \(
    2 \times 2 \times \cdots \times 2 = 2^k
    \)
    possible subsets. Think of bit-mask of set elements.
    % \\
    % \textbf{Example (bit-mask intuition for $\{a,b\}$):}
    % \(
    % {\scriptsize
    %         \begin{array}{ccl}
    %             00 & \longleftrightarrow & \varnothing \\
    %             01 & \longleftrightarrow & \{b\}       \\
    %             10 & \longleftrightarrow & \{a\}       \\
    %             11 & \longleftrightarrow & \{a,b\}
    %         \end{array}
    %     }
    % \)
    % \subsection*{Counting Principle}
    % If a process has:
    % \begin{itemize}[noitemsep, topsep=0pt]
    %     \item $a$ choices for step 1,
    %     \item $b$ choices for step 2,
    %     \item $c$ choices for step 3,
    % \end{itemize}
    % then the total number of outcomes is
    % \(
    % a \times b \times c.
    % \)
}
% \newline
