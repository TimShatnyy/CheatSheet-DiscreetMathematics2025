\subsection{Algebras, Monoids, Groups}
\tsDef{5.1 (Operations)}{
    Let $S$ be a set. A function
    \(
    \omega : S^n \to S \quad (n \ge 0)
    \)
    is called an \emph{operation} on $S$.
    \begin{itemize}[noitemsep, topsep=-1px]
        \item Arity $1$: unary operations
        \item Arity $2$: binary operations
        \item Arity $0$: constants
    \end{itemize}
}
\tsDef{5.2 (Algebra)}{
    An \emph{algebra} is a pair $\langle S, \Omega \rangle$, where $S$ is a set (also known as the \emph{carrier} of the algebra) and
    \(
    \Omega = (\omega_1, \dots, \omega_n)
    \)
    is a list of operations on $S$.
    \\
    Example:
    \(
    \langle \mathcal{P}(A), \cup, \cap, \rightarrow, \overline{\phantom{A}} \rangle
    \)
    where $\cup, \cap, \rightarrow$ are binary operations, and complement $( \ \overline{\phantom{A}} \ )$ is a unary operation.
}
\newline
\tsDef{5.3 (Neutral Element)}{
    A \emph{left/right neutral (identity) element} of an algebra $\langle S, * \rangle$ is an element $e \in S$ such that
    \(
    e * a = a \quad \text{or} \quad a * e = a \quad \text{for all } a \in S.
    \)
    If
    \(
    e * a = a * e = a \quad \text{for all } a \in S,
    \)
    then $e$ is simply called the \emph{neutral element}.
}
\newline
\tsLem{5.4 (Left-right neutral element equality)}{
    If $\langle S, * \rangle$ has both a left and a right neutral element, then they are equal.
    In particular, $\langle S, * \rangle$ has at most one neutral element.
}
\newline
\tsDef{5.4 (Associativity)}{
    A binary operation $*$ on a set $S$ is \emph{associative} if
    \(
    a * (b * c) = (a * b) * c \quad \text{for all } a,b,c \in S.
    \)
    \\
    This justifies the use of expressions such as
    \(
    \sum_{i=1}^n a_i \quad \text{or} \quad \prod_{i=1}^n a_i,
    \)
    since the order of addition or multiplication does not matter.
}
\newline
\tsDef{5.5 (Monoid)}{
    A \emph{monoid} is an algebra $\langle M, *, e \rangle$ where
    \begin{itemize}[noitemsep, topsep=-1px]
        \item $*$ is associative,
        \item $e$ is a neutral element.
    \end{itemize}
}
\tsDef{5.6 (Inverse)}{
    A left/right inverse of an element $a$ in an algebra $\langle S, *, e \rangle$ is an element $b \in S$ such that
    \(
    b * a = e \quad \text{or} \quad a * b = e.
    \)
    If
    \(
    b * a = a * b = e,
    \)
    then $b$ is called the \emph{inverse} of $a$.
}
\newline
\tsLem{5.7 (Left-right inverse leads to one inverse)}{
    In a monoid $\langle M, *, e \rangle$, if an element $a$ has both a left and a right inverse, then they are equal.
    In particular, $a$ has at most one inverse.
}
\newline
\tsDef{5.8 (Group)}{
    A \emph{group} is an algebra $\langle G, *,\widehat{\ }, e \rangle$ satisfying:
    \begin{itemize}[noitemsep, topsep=-1px]
        \item (G1) $*$ is associative,
        \item (G2) $e$ is a neutral element: $a * e = e * a = a$,
        \item (G3) $\forall a \in G$ there is an inverse $\widehat{a}$ such that
              \(
              a * \widehat{a} = \widehat{a} * a = e.
              \)
    \end{itemize}
}
% \newline
\tsLem{5.3 (Group properties)}{
    For a group $\langle G; *, \widehat{\ }, e \rangle$, we have for all $a,b,c \in G$:
    \begin{enumerate}[noitemsep, topsep=-1px]
        \item[(i)] $\widehat{(\widehat{a})} = a$.
        \item[(ii)] $\widehat{(a * b)} = \widehat{b} * \widehat{a}$.
        \item[(iii)] \textbf{Left cancellation law:}
              \(
              a * b = a * c \;\Longrightarrow\; b = c.
              \)
        \item[(iv)] \textbf{Right cancellation law:}
              \(
              b * a = c * a \;\Longrightarrow\; b = c.
              \)
        \item[(v)] The equation $a * x = b$ has a unique solution $x$ for any $a$ and $b$.
              So does the equation $x * a = b$.
    \end{enumerate}
}
% \newline
\tsDef{5.9 (Abelian/Commutative Group)}{
    Group $\langle G, * \rangle$ is called \emph{abelian} (commut.) if
    \(
    a * b = b * a \ \forall a,b \in G.
    \)
}
