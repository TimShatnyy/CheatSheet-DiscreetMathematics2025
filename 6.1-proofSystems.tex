\subsection{Proof Systems}
\tsDef{6.1 (Proof System)}{
    A proof system is a quadruple
    \(
    \Pi = (\mathcal{S}, \mathcal{P}, \tau,\phi)
    \)
    where:
    \begin{itemize}[noitemsep, topsep=0px]
        \item $\mathcal{S}$ is a set of statements,
        \item $\mathcal{P}$ is a set of proofs,
        \item $\tau : \mathcal{S} \to \{0,1\}$ is a truth function,
        \item $\varphi : \mathcal{S} \times \mathcal{P} \to \{0,1\}$ is a verification function.
    \end{itemize}
    A proof $p \in \mathcal{P}$ is \emph{valid} for a statement $s \in S$ if $\varphi(s,p)=1$.
    A valid proof means it proves the statement.
}
\newline
\tsDef{6.2 (Soundness)}{
    A proof system is \emph{sound} if no false statement has a proof:
    \(
    \forall s \in \mathcal{S},\; \exists p \in \mathcal{P} \text{ with } \varphi(s,p)=1 \;\Rightarrow\; T(s)=1.
    \)
}
\newline
\tsDef{6.3 (Completeness)}{
    A proof system is \emph{complete} if every true statement has a proof:
    \(
    \forall s \in \mathcal{S},\; T(s)=1 \;\Rightarrow\; \exists p \in \mathcal{P} \text{ with } \varphi(s,p)=1.
    \)
}
\newline
\tsDef{6.4 (Syntax and Semantics)}{
    The \emph{syntax} of a logic defines an alphabet $\Lambda$ of allowed symbols and specifies which strings in $\Lambda^*$ are well-formed formulas.
    \\
    The \emph{semantics} describes under which conditions a formula is true ($1$) or false ($0$).
    Syntax concerns form; semantics concerns meaning.
    \\
    Different syntactic expressions may have the same semantics, e.g.
    \(
    i := i+1 \quad \text{and} \quad i+ = 1.
    \)
}
\newline
\tsDef{6.5 (Free Variables)}{
    The semantics of a logic assigns to each formula $F=(f_1,f_2,\dots,f_k) \in \Lambda^*$ a subset
    \(
    \mathrm{free}(F) \subseteq \{f_1,\dots,f_k\}
    \)
    of indices. If $i \in \mathrm{free}(F)$, then symbol $f_i$ occurs free in $F$.
}
\newline
\tsDef{6.6 (Interpretation)}{
    An interpretation consists of:
    \(\tsPoint\) a set $Z \subseteq \Lambda$ of symbols,
    \(\tsPoint\) a domain (a set of possible values),
    \(\tsPoint\) a function assigning to each symbol in $Z$ a value in the domain.
}
\newline
\tsDef{6.7 (Suitable Interpretation)}{
    A suitable interpretation assigns a value to all symbols $p \in \Lambda$ occurring free in a formula $F$.
}
\newline
\tsDef{6.8 (Truth Value)}{
    The semantics of a logic defines a function assigning to each formula $F$ and each suitable interpretation $\mathcal{A}$ a truth value
    \(
    \mathcal{A}(F) \in \{0,1\}.
    \)
    We write $\mathcal{A}(F)$ for the truth value of $F$ under interpretation $\mathcal{A}$.
}
\newline
\tsDef{6.9 (Model)}{
    A suitable interpretation $\mathcal{A}$ for which a formula $F$ is true,
    \(
    \mathcal{A}(F)=1,
    \)
    is called a \emph{model} of $F$, written $\mathcal{A} \models F$.
    \\
    If $\mathcal{A}$ is a model for all formulas in a set $M$, we write $\mathcal{A} \models M$.
}
\newline
\tsDef{6.10 (Satisfiability)}{
    A formula $F$ is \emph{satisfiable} if it has a model.
    It is \emph{unsatisfiable} otherwise.
    The symbol $\bot$ denotes an unsatisfiable formula.
}
\newline
\tsDef{6.11 (Tautology)}{
    A formula $F$ is a \emph{tautology} if it is true under every suitable interpretation.
    The symbol $\top$ denotes a tautology.
}
\newline
\tsDef{6.12 (Logical Consequence)}{
    Let $F$ be a set of formulas and $G$ a formula.
    We say $G$ is a \emph{logical consequence} of $F$, written
    \(
    F \models G,
    \)
    if every interpretation that is a model of $F$ is also a model of $G$.
}
\newline
\tsDef{6.13 (Logical Equivalence)}{
    Formulas $F$ and $G$ are \emph{logically equivalent}, written $F \equiv G$, if
    \(
    F \models G \ \text{and} \ G \models F.
    \)
}
\newline
\tsDef{6.14}{
    A formula $F$ is a tautology iff
    \(
    \models F.
    \)
    A formula $F$ is unsatisfiable iff
    \(
    F \equiv \bot.
    \)
}
\newline
\tsLem{6.2 (Formula is tautology iff neg. unsatisfiable)}{
    $F$ is a tautology if and only if $\neg F$ is unsatisfiable.
}
\newline
\tsLem{6.3 (Statements to prove the unsat. of formulas)}{
    $\{F_1,\dots,F_n\} \models G$ if and only if
    \(
    (F_1 \land \dots \land F_n) \to G
    \)
    is a tautology and if and only if
    $\{F_1, ..., F_k, \neg G\}$. Statement are equivalent.
}
\newline
