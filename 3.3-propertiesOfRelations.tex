\subsection{Properties of Relations}
\begin{center}
    \renewcommand{\arraystretch}{1.4}
    \begin{tabular}{lccc}
        \toprule
        \textbf{Name} & \textbf{Formula}                             & \textbf{Set} & \textbf{Example} \\
        \midrule
        Reflexive
                      & $a\rho a$
                      & $\mathrm{id} \subseteq \rho$
                      & $\mathrm{id},\, \geq$                                                          \\

        Irreflexive
                      & $\neg(a\rho a)$
                      & $\mathrm{id} \cap \rho = \varnothing$
                      & $\neq,\, >$                                                                    \\

        Symmetric
                      & $a\rho b \iff b\rho a$
                      & $\rho = \hat{\rho}$
                      & $\mathrm{id},\, \equiv\ (\mathrm{mod}\ m)$                                     \\

        Antisymmetric
                      & $a\rho b \wedge b\rho a \rightarrow a=b$
                      & $\rho \cap \hat{\rho} \subseteq \mathrm{id}$
                      & $\geq,\, \mid$                                                                 \\

        Transitive
                      & $a\rho b \wedge b\rho c \rightarrow a\rho c$
                      & $\rho^{2} \subseteq \rho$
                      & $\equiv\ (\mathrm{mod}\ m),\, >$                                               \\
        \bottomrule
    \end{tabular}
\end{center}
% \tsDef{3.13 (Reflexive Relation)}{
%     A relation $\rho$ on a set $A$ is called \emph{reflexive} if
%     \(
%     \forall a \in A,\ a \,\rho\, a.
%     \)
%     \\
%     Equivalently,
%     \(
%     \mathrm{id}_A \subseteq \rho.
%     \)
% }
% \newline
% \tsDef{3.14 (Irreflexive Relation)}{
%     A relation $\rho$ on a set $A$ is called \emph{irreflexive} if
%     \(
%     \forall a \in A,\ a \not\!\rho\, a.
%     \)
%     \\
%     Equivalently,
%     \(
%     \rho \cap \mathrm{id}_A = \varnothing.
%     \)
% }
% \newline
% \tsDef{3.15 (Symmetric Relation)}{
%     A relation $\rho$ on a set $A$ is called \emph{symmetric} if
%     \(
%     a \,\rho\, b \;\Leftrightarrow\; b \,\rho\, a
%     \ \forall a,b \in A.
%     \)
%     \\
%     Equivalently,
%     \(
%     \rho = \rho^{-1}.
%     \)
%     \\
%     Interpretations:
%     \begin{itemize}[noitemsep, topsep=-3pt]
%         \item In matrix form: the matrix of $\rho$ is symmetric.
%         \item In graphs: the graph is undirected (possibly with loops).
%     \end{itemize}
% }
% \tsDef{3.16 (Antisymmetric Relation)}{
%     A relation $\rho$ on a set $A$ is called \emph{antisymmetric} if
%     \(
%     (a \,\rho\, b \;\wedge\; b \,\rho\, a) \;\Rightarrow\; a = b
%     \ \forall a,b \in A.
%     \)
%     \\
%     Equivalently,
%     \(
%     \rho \cap \rho^{-1} \subseteq \mathrm{id}_A.
%     \)
% }
% \newline
% \tsDef{3.17 (Transitive Relation)}{
%     A relation $\rho$ on a set $A$ is called \emph{transitive} if
%     \(
%     (a \,\rho\, b \;\wedge\; b \,\rho\, c) \;\Longrightarrow\; a \,\rho\, c.
%     \)
% }
% \newline
% \newline
\tsLem{3.9 (Transitivity and relation composition)}{
    A relation $\rho$ is transitive if and only if
    \(
    \rho^2 \subseteq \rho,
    \)
    where $\rho^2 = \rho \circ \rho$.
}
\newline
\tsDef{3.18 (Transitive Closure)}{
    The \emph{transitive closure} of a relation $\rho$ on a set $A$, denoted $\rho^{*}$, is defined by
    \(
    \rho^{*} := \bigcup_{n \in \mathbb{N}_{>0}} \rho^{n}.
    \)
    For a transitive relation \(\rho\) we have \(\rho^2 \subseteq \rho\).
}
\newline
