\subsection*{\textit{Algebra}}
\tsIdea{Finding number of subgroups}{
    The number of subgroups of $\mathbb{Z}_m \times \mathbb{Z}_n$ is given by
    \(
    \sum_{\substack{a \mid m \\ b \mid n}} \gcd(a,b).
    \)
}
\newline
\tsIdea{Generators of max. cyclic subgroups and their order}{
    Compute the group orders of $\mathbb{Z}_m^*$ and $\mathbb{Z}_n^*$ separately.
    $\tsPoint$ For finite groups $G_1$ and $G_2$ and $(a,b)\in G_1\times G_2$, \\
    \(
    \operatorname{ord}(a,b)
    = \mathrm{lcm}\bigl(\operatorname{ord}(a),\operatorname{ord}(b)\bigr).
    \)
    $\tsPoint$ Choose elements of maximal order in each factor to obtain a generator
    of a largest cyclic subgroup.
}
\newline
\tsIdea{Finding isomoprhic groups}{
    Testing the equality between multiplicative group orders with $\varphi(m)$ is not sufficient, it's also necessary to check cyclicity to preserve the group structure.
    \\
    $\mathbb{Z}^*_{8}$ and $\mathbb{Z}^*_{12}$ are not cyclic: the group $\mathbb{Z}^*_m \text{ is cyclic } \iff m \in \{2, 4, p^e, 2p^e\}$ where $p$ is an odd prime and $e \geq 1$ exponent (Th 5.15).
}
\newline
\tsIdea{Finding GCD of 2 polynomials}{
    \(
    \text{Let } f_0(x),\, f_1(x) \in \mathbb{F}[x], \quad \deg f_0 \ge \deg f_1.
    \)
    \\
    \(
    f_0(x) = q_1(x)f_1(x) + r_1(x), \quad \deg r_1 < \deg f_1
    \)
    \\
    \(
    f_1(x) = q_2(x)r_1(x) + r_2(x), \quad \deg r_2 < \deg r_1
    \)
    \\
    \(
    r_1(x) = q_3(x)r_2(x) + r_3(x)
    \)
    \\
    \(
    \vdots
    \)
    \\
    \(
    r_{k-2}(x) = q_k(x)r_{k-1}(x) + r_k(x)
    \)
    \\
    \(
    r_{k-1}(x) = q_{k+1}(x)r_k(x)
    \)
    \\
    \(
    \boxed{\gcd(f_0(x),f_1(x)) = r_k(x)}
    \)
    \\
    (Optional: normalize $r_k(x)$ to be monic. Take out factors by finding common roots for finding divisors of both polynomials, which after division will be inserted in the next iteration of gcd.)
}
\newline
\tsIdea{there exist infinitely many irreducible polynomials in $F[x]$}{
    Suppose, towards a contradiction, that there are only finitely many irreducible
    polynomials in $F[x]$, say
    \(
    f_1(x), f_2(x), \dots, f_n(x).
    \)

    Since $F$ is a field, $F[x]$ is a Euclidean domain, and hence every non-constant
    polynomial factors into irreducible polynomials.

    Define
    \(
    p(x) = f_1(x) f_2(x) \cdots f_n(x) + 1.
    \)
    Then $p(x) \in F[x]$ and $\deg p(x) \ge 1$.

    For each $i \in \{1,\dots,n\}$, we have
    \(
    p(x) \equiv 1 \pmod{f_i(x)},
    \)
    so $f_i(x)$ does not divide $p(x)$. Therefore, $p(x)$ has no irreducible factor
    among $f_1(x),\dots,f_n(x)$.

    If $p(x)$ is irreducible, then it is an irreducible polynomial not in the list,
    a contradiction. If $p(x)$ is reducible, then it factors into irreducible
    polynomials, none of which can be among $f_1(x),\dots,f_n(x)$, again a
    contradiction.

    Hence, there must be infinitely many irreducible polynomials in $F[x]$.
}
\tsIdea{Proving a field with polynomials}{
Example: Show \(\mathbb{Z}_3[x]_{x^4+x+2}\) is a field:
\(\tsPoint\) show that \(\mathbb{Z}_3\) is a field.
\(\tsPoint\) show that \(x^4+x+2\) is irreducible.
}
