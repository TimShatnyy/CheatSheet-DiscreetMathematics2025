% \subsection{Number Theory}
\tsDef{4.1}{
    Let $a,b \in \mathbb{Z}$.
    We say that $a$ \emph{divides} $b$, written $\bullet \ a \mid b$, if there exists $c \in \mathbb{Z}$ such that
    \(
    \bullet \ b = ac.
    \)
    If $a \neq 0$, then this quotient is unique and
    \(
    c = \frac{b}{a}.
    \)
    Every nonzero integer divides $0$.
    The integers $1$ and $-1$ divide every integer.
}
\newline
\tsThe{4.1 (Euclid)}{
    For all integers $a$ and $d \neq 0$, there exist unique integers $q$ and $r$ such that:
    \(
    \bullet \ a = dq + r \ \text{and} \ 0 \le r < |d|.
    \)
    \\
    Here: $d$ - \emph{divisor}, $a$ - \emph{dividend}, $q$ - \emph{quotient}, $r$ - \emph{remainder}.
    % \begin{itemize}[noitemsep, topsep=0pt]
    %     \item $d$ is the \emph{divisor},
    %     \item $a$ is the \emph{dividend},
    %     \item $q$ is the \emph{quotient},
    %     \item $r$ is the \emph{remainder}.
    % \end{itemize}
    \\
    The remainder is denoted by
    \(
    \bullet \ r = R_d(a) \ \text{or} \ \bullet \ r = a \bmod d.
    \)
}
\newline
\tsDef{4.2 (Greatest Common Divisor)}{
    The \emph{greatest common divisor} of $a$ and $b$, denoted $\gcd(a,b)$, is the integer $d$ such that
    \(
    \bullet \
    d \mid a \;\wedge\; d \mid b \;\wedge\;
    \bigl(\forall c \, (c \mid a \wedge c \mid b \Rightarrow c \mid d)\bigr).
    \)
}
\newline
\tsLem{4.3}{
    If $a,b \in \mathbb{Z}$ are \textit{relatively prime}, then
    \(
    \bullet \ \gcd(a,b) = 1.
    \)
}
\newline
\tsLem{4.2}{
    For all $m,n,q \in \mathbb{Z}$,
    \(
    \bullet \
    \gcd(m, n + qm) = \gcd(m,n).
    \)
    In particular,
    \(
    \bullet \ \gcd\bigl(m, R_m(n)\bigr) = \gcd(m,n),
    \)
    which is the basis of the Euclidean algorithm.
}
\newline
\tsDef{4.4}{
    For $a,b \in \mathbb{Z}$, the \emph{ideal generated by $a$ and $b$}, denoted $(a,b)$, is defined by
    \(
    (a,b) = \{ ua + vb \mid u,v \in \mathbb{Z} \}.
    \)
    For a single integer $a$, the ideal generated by $a$ is
    \(
    (a) = \{ ua \mid u \in \mathbb{Z} \}.
    \)
    Every ideal in $\mathbb{Z}$ can be generated by a single integer.
}
\newline
\tsLem{4.3}{
    For $a,b \in \mathbb{Z}$, there exists $d \in \mathbb{Z}$ such that
    \(
    (a,b) = (d).
    \)
}
\newline
\tsLem{4.4}{
    Let $a,b \in \mathbb{Z}$, not both zero.
    If $(a,b) = (d)$, then $d$ is a greatest common divisor of $a$ and $b$.
}
\newline
\tsCor{4.5}{
    If $a \in \mathbb{Z}$, then
    \(
    (a,0) = (a).
    \)
}
\newline
\tsCor{(Bézout)}{
    For $a,b \in \mathbb{Z}$, not both zero, there exist $u,v \in \mathbb{Z}$ such that
    \(
    \gcd(a,b) = ua + vb.
    \)
    \\
    Example:
    \(
    \gcd(26,18) = 2 = (-2)\cdot 26 + 3\cdot 18.
    \)
}
\newline
\tsDef{4.5 (Least Common Multiple)}{
    The \emph{least common multiple} $\ell$ of positive integers $a$ and $b$ is the integer satisfying
    \(
    a \mid \ell \;\wedge\; b \mid \ell \;\wedge\;
    \bigl(\forall m \, (a \mid m \wedge b \mid m \Rightarrow \ell \mid m)\bigr).
    \)
}
\newline
\tsDef{4.6}{
    A positive integer $p > 1$ is called \emph{prime} if the only positive divisors of $p$ are $1$ and $p$.
    An integer greater than $1$ that is not prime is called \emph{composite}.
}
\newline
\tsThe{4.6 (Fundamental Theorem of Arithmetic)}{
    Every positive integer can be written uniquely (up to the order in which the factors are listed) as a product of primes.
    \\
    Thus, if
    \(
    a = \prod_i p_i^{e_i}
    \ \text{and} \
    b = \prod_i p_i^{f_i},
    \)
    then
    \(
    \gcd(a,b) = \prod_i p_i^{\min(e_i,f_i)},
    \)
    and
    \(
    \operatorname{lcm}(a,b) = \prod_i p_i^{\max(e_i,f_i)}.
    \)
    In particular,
    \(
    \gcd(a,b)\cdot \operatorname{lcm}(a,b) = ab,
    \)
    since
    \(
    \min(e_i,f_i) + \max(e_i,f_i) = e_i + f_i.
    \)
}
\newline
\tsLem{4.7}{
    Every composite integer $n$ has a prime divisor $p \le \sqrt{n}$.
}
\newline
\tsDef{4.8 (Congruences)}{
    For $a,b,m \in \mathbb{Z}$ with $m \ge 1$, we say that $a$ is \emph{congruent} to $b$ modulo $m$ if $m$ divides $a-b$.
    We write
    \(
    a \equiv b \pmod{m},
    \)
    or simply
    \(
    a \equiv_m b.
    \)
    Equivalently,
    \(
    a \equiv_m b \;\Longleftrightarrow\; m \mid (a-b).
    \)
}
\newline
\tsLem{4.13}{
    For any $m \ge 1$, the relation $\equiv_m$ is an equivalence relation on $\mathbb{Z}$.
}
\newline
\tsLem{4.14}{
    If $a \equiv_m b$ and $c \equiv_m d$, then
    \(
    a + c \equiv_m b + d
    \ \text{and} \
    ac \equiv_m bd.
    \)
}
\newline
\tsCor{4.15}{
    Let $f(x_1,\dots,x_k)$ be a multivariable polynomial in $k$ variables with integer coefficients, and let $m \ge 1$.
    If
    \(
    a_i \equiv_m b_i \quad \text{for } 1 \le i \le k,
    \)
    then
    \(
    f(a_1,\dots,a_k) \equiv_m f(b_1,\dots,b_k).
    \)
}
\newline
\tsLem{4.16}{
    For all $a,b,m \in \mathbb{Z}$ with $m \ge 1$:
    \begin{enumerate}[noitemsep, topsep=0pt]
        \item[(i)] $a \equiv_m R_m(a)$.
        \item[(ii)] $a \equiv_m b \;\Longleftrightarrow\; R_m(a) = R_m(b)$.
    \end{enumerate}
}
\tsCor{4.17}{
    Let $f(x_1,\dots,x_k)$ be a multivariable polynomial with integer coefficients, and let $m \ge 1$, then:
    \\
    \(
    R_m\!\bigl(f(a_1,\dots,a_k)\bigr)
    =
    R_m\!\bigl(f(R_m(a_1),\dots,R_m(a_k))\bigr).
    \)
}
\newline
\tsThe{4.18}{
    The congruence equation
    \(
    ax \equiv_m 1
    \)
    has a solution $x \in \mathbb{Z}_m$ if and only if
    \(
    \gcd(a,m) = 1.
    \)
    In this case, the solution is unique.
}
\newline
\tsDef{4.9}{
    If $\gcd(a,m)=1$, the unique solution $x \in \mathbb{Z}_m$ to the congruence equation
    \(
    ax \equiv_m 1
    \)
    is called the \emph{multiplicative inverse} of $a$ modulo $m$.
    \\
    Other notation:
    \(
    x \equiv_m a^{-1}
    \ \text{or} \
    x \equiv_m \frac{1}{a}.
    \)
    The multiplicative inverse can be efficiently computed using the \emph{extended Euclidean algorithm}.
}
\newline
\tsThe{4.10 (Chinese Remainder Theorem)}{
    Let $m_1,m_2,\dots,m_r$ be pairwise relatively prime integers, and let
    \(
    M = \prod_{i=1}^r m_i.
    \)
    For every list of integers $a_1,\dots,a_r$ with
    \(
    0 \le a_i < m_i \quad \text{for } 1 \le i \le r,
    \)
    the system of congruences
    \(
    (
    x \equiv_{m_1} a_1, \
    x \equiv_{m_2} a_2,
    \ \dots \ , \
    x \equiv_{m_r} a_r
    )
    \)
    has a unique solution $x$ satisfying
    \(
    0 \le x < M.
    \)
}