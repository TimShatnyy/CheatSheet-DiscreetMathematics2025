\subsection{Relations}
\tsDef{3.8 (Cartesian product)}{
    The Cartesian product $A \times B$ of sets $A$ and $B$ is the set of all ordered pairs with first component from $A$ and second from $B$:
    \(
    A \times B := \{(a,b) \mid a \in A,\ b \in B\}.
    \)
    The cardinality satisfies
    \(
    |A \times B| = |A| \cdot |B|.
    \)
    \\
    \textbf{More generally:}
    \(
    \mathop{\scalebox{1.6}{$\times$}}_{i=1}^k A_i
    :=
    \{(a_1,\dots,a_k) \mid a_i \in A_i \text{ for } 1 \le i \le k\}.
    \)
    \\
    The Cartesian product is \emph{not associative}, since elements are ordered tuples.
    % \\
    % \textbf{Connection to power sets:}
    % \\
    % If
    % \(
    % A = \{a,b,c\}, \ |A|=3,
    % \)
    % then each element may be either \emph{in} or \emph{out}, giving
    % \(
    % \{0,1\}^3 = \{0,1\} \times \{0,1\} \times \{0,1\},
    % \)
    % which represents all subsets of $A$.
    \\
    \textbf{Example:}
    \\
    \(
    A_1 = \{0,1\}, \ A_2 = \{d,e\},
    \)
    \(
    A_1 \times A_2 = \{(0,d),(0,e),(1,d),(1,e)\}.
    \)
}
\newline
\tsDef{3.9 (Relation)}{
    A (binary) relation $\rho$ from a set $A$ to a set $B$ is a subset of $A \times B$.
    \\
    If $A=B$, then $\rho$ is called a relation \emph{on} $A$.
    \\
    \textbf{Notation:}
    \(
    (a,b) \in \rho \;\Rightarrow\; a \,\rho\, b,
    \
    (a,b) \notin \rho \;\Rightarrow\; a \not\!\rho\, b.
    \)
    \\
    For any set $S$, any subset $\rho \subseteq S \times S$ is a relation on $S$.
    \\
    There are
    \(
    2^{n^2}
    \)
    relations on a set of cardinality $n$, since
    \(
    |S \times S| = n^2
    \ \text{and} \
    |\mathcal{P}(S \times S)| = 2^{n^2}.
    \)
    \\
    % \textbf{Relations as Matrices:}
    % \\
    % Relations can be represented by $0$–$1$ matrices, analogous to adjacency matrices for graphs. Combining relations using set operations
    % \(
    % \cap,\ \cup,\ \setminus
    % \)
    % corresponds to applying logical operations
    % \(
    % \wedge,\ \vee,\ \neg
    % \)
    % entrywise to matrices.
    % \\
    \textbf{Examples on $\mathbb{Z}$:}
    \begin{itemize}[noitemsep, topsep=-3pt]
        \item $\leq \;\cup\; \geq$ is the complete relation $\mathbb{Z} \times \mathbb{Z}$.
        \item $\leq \;\cap\; \geq$ is the identity relation:
              \(
              \{(a,a) \mid a \in \mathbb{Z}\}.
              \)
    \end{itemize}
}
\tsDef{3.11 (Inverse Relation)}{
    The inverse relation of $\rho$ is
    \(
    \rho^{-1} := \{(b,a) \mid (a,b) \in \rho\}.
    \)
    \\
    Equivalently,
    \(
    b \,\rho^{-1}\, a \;\Longleftrightarrow\; a \,\rho\, b.
    \)
    \\
    % \textbf{Interpretations:}
    % \begin{itemize}[noitemsep, topsep=-3pt]
    %     \item In graphs: reversing all edge directions.
    %     \item In matrices: taking the transpose.
    % \end{itemize}
}
\tsDef{3.12 (Composition of Relations)}{
    Let $\rho \subseteq A \times B$ and $\sigma \subseteq B \times C$.
    The composition $\sigma \circ \rho$ is defined by
    \(
    \sigma \circ \rho
    :=
    \{(a,c) \mid \exists b \ ((a,b) \in \rho \wedge (b,c) \in \sigma)\}.
    \)
    Composition is associative:
    \(
    \rho \circ (\sigma \circ \tau) = (\rho \circ \sigma) \circ \tau.
    \)
}
\newline
\tsLem{3.8 (Inverse of relation composition)}{
    Let $\rho$ be a relation from $A$ to $B$ and $\sigma$ a relation from $B$ to $C$.
    Then
    \(
    (\sigma \circ \rho)^{-1} = \rho^{-1} \circ \sigma^{-1}.
    \)
}
