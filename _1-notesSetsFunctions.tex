\subsection*{\textit{Sets, relations}}
\tsIdea{Finding elements/subsets}{
    List elements of A explicily as a bullet list.
    \(\tsPoint\) \(x \in A\): candidate set is exactly equal to one of the listed elements.
    \(\tsPoint\) \(y \subseteq A\): every element of \(y\) is also in \(A\).
    \(\tsPoint\) \(\emptyset\) is a subset of everyset, \(\{\emptyset\}\) has 1 element of emptyset.
}
\newline
\tsIdea{Simplification of set expressions}{
    Draw Venn diagramm.
    \(\tsPoint\) Convert expression to boolean logic: \\ \(p,q,r := x \in A,B,C\), then \((A \cap B) \setminus B =  (p \land q) \land \lnot(r)\).
}
\newline
\tsIdea{Working with cadinalities of sets and combinations}{
    \(|A \times B| = |A| \cdot |B|\)
    \(\tsPoint\) If \(A \cap B = \emptyset\), then \(|A \cup B| = |A| + |B|\), as they are disjoint.
    \(\tsPoint\) Define and split in cases.
}
\newline
\tsIdea{Transitivity and Symmetry forces reflexivity}{
    Transitivity and Symmetry forces reflexivity: \((a,b), (b,a) \in \rho, \ \rho = \hat{\rho} \Rightarrow (a,a) \in \rho\).
}
\newline
\tsIdea{Upper bounds}{
    Find lcm of a given set (divisibility) and find multiples of it which are greater than set elements.
    \(\tsPoint\) If we want in divisibility relation least upper bound, then find the upper bound that divides all other upper bounds. Otherwise it doesn't exist.
}
\newline
\tsIdea{Proving function existence}{
    First define if it is totally and well defined function. Then work with given condiitions.
}